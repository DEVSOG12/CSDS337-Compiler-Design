\documentclass[a4paper, 11pt]{article}
\usepackage{comment} % enables the use of multi-line comments (\ifx \fi) 
\usepackage{fullpage} % changes the margin
\usepackage[a4paper, total={7in, 10in}]{geometry}
\usepackage[fleqn]{amsmath}
\usepackage{amssymb,amsthm}  % assumes amsmath package installed
\newtheorem{theorem}{Theorem}
\newtheorem{corollary}{Corollary}
\usepackage{graphicx}
\usepackage{tikz}
\usetikzlibrary{arrows}
\usepackage{verbatim}
\usepackage[numbered]{mcode}
\usepackage{float}
\usepackage{tikz}
    \usetikzlibrary{shapes,arrows}
    \usetikzlibrary{arrows,calc,positioning}

    \tikzset{
        block/.style = {draw, rectangle,
            minimum height=1cm,
            minimum width=1.5cm},
        input/.style = {coordinate,node distance=1cm},
        output/.style = {coordinate,node distance=4cm},
        arrow/.style={draw, -latex,node distance=2cm},
        pinstyle/.style = {pin edge={latex-, black,node distance=2cm}},
        sum/.style = {draw, circle, node distance=1cm},
    }
\usepackage{xcolor}
\usepackage{mdframed}
\usepackage[shortlabels]{enumitem}
\usepackage{indentfirst}
\usepackage{hyperref}
    
\renewcommand{\thesubsection}{\thesection.\alph{subsection}}

\newenvironment{problem}[2][Problem]
    { \begin{mdframed}[backgroundcolor=gray!20] \textbf{#1 #2} \\}
    {  \end{mdframed}}

% Define solution environment
\newenvironment{solution}
    {\textit{Solution:}}
    {}

\renewcommand{\qed}{\quad\qedsymbol}
%%%%%%%%%%%%%%%%%%%%%%%%%%%%%%%%%%%%%%%%%%%%%%%%%%%%%%%%%%%%%%%%%%%%%%%%%%%%%%%%%%%%%%%%%%%%%%%%%%%%%%%%%%%%%%%%%%%%%%%%%%%%%%%%%%%%%%%%
\begin{document}
%Header-Make sure you update this information!!!!
\noindent
%%%%%%%%%%%%%%%%%%%%%%%%%%%%%%%%%%%%%%%%%%%%%%%%%%%%%%%%%%%%%%%%%%%%%%%%%%%%%%%%%%%%%%%%%%%%%%%%%%%%%%%%%%%%%%%%%%%%%%%%%%%%%%%%%%%%%%%%
\large\textbf{Your name} \hfill \textbf{Homework - \#}   \\
Email: youremail@case.edu \hfill ID: 123456789 \\
\normalsize Course: CSDS 337 - Compiler Design \hfill Term: Spring 2024\\
Instructor: Dr. Vipin Chaudhary \hfill Due Date: $20^{th}$ March, 2024 \\ \\
Number of hours delay for this Problem Set: \hfill Put hours here\\
Cumulative number of hours delay so far: \hfill Put hours here \\ \\
I discussed this homework with: \hfill Put names here \\
\noindent\rule{7in}{2.8pt}

\large\textbf{More information is available in the README.md file. } \\

\large\textbf{An Intro To LLVM} \\
LLVM is a powerful framework that allows you to generate and even execute code. In this assignment, we will look at the basics of programming with LLVM. This is by no means a complete guide, but our hope is that you get an understanding on how LLVM works and can use this as a base for future LLVM-related projects. \\

\large\textbf{Assignment} \\
Your assignment is to modify main.cpp to generate code for 6 optimized functions.
Each stage of the assignment will get progressively harder (though sometimes getting started can be the hardest).
Please see the guide section of the README first for a quick guide about LLVM and for common FAQs.
WINDOWS\_SETUP instructions are provided, though the VM will have all the packages needed by default.
A runDocker.sh script is provided for building with docker if that is preferred.
You may need to `chmod +x run.sh` and other scripts first before running them with `./run.sh`. \\

\begin{problem}{1}
Before we can do anything, an LLVM context and module needs to be establed. This module should be called `sampleMod`. Have this module be printed (printing will print LLVM assembly) to `sampleMod.ll` and have its bitcode be written to `sampleMod.bc`. Note that the printed `.ll` file is text, and will show you the exact code you are generating.

\end{problem}

\begin{problem}{2}
Create a function named `simple`. It takes no parameters, and returns a 32-bit integer. Have it return 0.
Use `llvm::verifyFunction(*FUNC\_HERE);` after you are done building a function to check for errors.

\end{problem}

\begin{problem}{3}
Create a function named `add` that takes two 32-bit integers and returns the sum of them (a 32-bit integer).

\end{problem}

\begin{problem}{4}
Create a function named `addIntFloat` that has the first parameter be a 32-bit integer and the second parameter be a float. The function should return a float. Note that there are different types of casts to float, you can assume the input integer is signed.

\end{problem}

\begin{problem}{5}
Adding things together in a linear fashion is fun, but what about temporary variables and control flow? Create a function called `conditional`, it will take a boolean input (1-bit integer) and output a 32-bit integer. Allocate a mutable variable stored on the stack in the entry block. If the input parameter is true, store a `3` to the variable, else store a `5`. Using only one add instruction in the entire function, return the value of the stored variable added with `11`. The function should return `14` if the parameter is true or `16` if it is false. **DO NOT OPTIMIZE THIS FUNCTION YOURSELF.** The point of this part is to make sure you understand control flow and mutable stack variables.

\end{problem}

\begin{problem}{6}
Do the last part again, except name the function `oneTwoPhi` and do it with phi nodes instead.

\end{problem}

\begin{problem}{7}
Do the last part again, except name the function `selection` and do it with the `select` instruction instead.

\end{problem}

\begin{problem}{8}
Add an LLVM legacy function pass manager `llvm::legacy::FunctionPassManager`. Add the `llvm::createPromoteMemoryToRegisterPass()`, `llvm::createReassociatePass()`, `llvm::createGVNPass()`, and `llvm::createCFGSimplificationPass()` passes and run it on the 6 functions you created earlier. What was the effect of these passes on each of the 6 functions?

\end{problem}

\begin{problem}{9}
By now, you should have a decent base knowledge of generating code with LLVM. Note that the below items are completely optional, but they can give you ideas on what you can do: \\
* Make hello world in LLVM. \\
* Make a function's arguments mutable local to the function by storing them into stack variables. \\
* Declare the `malloc` and `free` functions and call them in different functions to play with heap memory. \\
* Make a while or for loop with LLVM. \\
* Play around with struct types. \\
* Create a basic calculator interpreter program. \\

\end{problem}

%\lstinputlisting{SampleCode.m}
\noindent\rule{7in}{2.8pt}
%%%%%%%%%%%%%%%%%%%%%%%%%%%%%%%%%%%%%%%%%%%%%%%%%%%%%%%%%%%%%%%%%%%%%%%%%
\end{document}
 