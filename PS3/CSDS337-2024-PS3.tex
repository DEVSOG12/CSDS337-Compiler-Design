\documentclass[a4paper, 11pt]{article}
\usepackage{comment} % enables the use of multi-line comments (\ifx \fi) 
\usepackage{lipsum} %This package just generates Lorem Ipsum filler text. 
\usepackage{fullpage} % changes the margin
\usepackage[a4paper, total={7in, 10in}]{geometry}
\usepackage[fleqn]{amsmath}
\usepackage{amssymb,amsthm}  % assumes amsmath package installed
\newtheorem{theorem}{Theorem}
\newtheorem{corollary}{Corollary}
\usepackage{graphicx}
\usepackage{tikz}
\usetikzlibrary{arrows}
\usepackage{verbatim}
\usepackage[numbered]{mcode}
\usepackage{float}
\usepackage{tikz}
    \usetikzlibrary{shapes,arrows}
    \usetikzlibrary{arrows,calc,positioning}

    \tikzset{
        block/.style = {draw, rectangle,
            minimum height=1cm,
            minimum width=1.5cm},
        input/.style = {coordinate,node distance=1cm},
        output/.style = {coordinate,node distance=4cm},
        arrow/.style={draw, -latex,node distance=2cm},
        pinstyle/.style = {pin edge={latex-, black,node distance=2cm}},
        sum/.style = {draw, circle, node distance=1cm},
    }
\usepackage{xcolor}
\usepackage{mdframed}
\usepackage[shortlabels]{enumitem}
\usepackage{indentfirst}
\usepackage{hyperref}
    
\renewcommand{\thesubsection}{\thesection.\alph{subsection}}

\newenvironment{problem}[2][Problem]
    { \begin{mdframed}[backgroundcolor=gray!20] \textbf{#1 #2} \\}
    {  \end{mdframed}}

% Define solution environment
\newenvironment{solution}
    {\textit{Solution:}}
    {}

\renewcommand{\qed}{\quad\qedsymbol}
%%%%%%%%%%%%%%%%%%%%%%%%%%%%%%%%%%%%%%%%%%%%%%%%%%%%%%%%%%%%%%%%%%%%%%%%%%%%%%%%%%%%%%%%%%%%%%%%%%%%%%%%%%%%%%%%%%%%%%%%%%%%%%%%%%%%%%%%
\begin{document}
%Header-Make sure you update this information!!!!
\noindent
%%%%%%%%%%%%%%%%%%%%%%%%%%%%%%%%%%%%%%%%%%%%%%%%%%%%%%%%%%%%%%%%%%%%%%%%%%%%%%%%%%%%%%%%%%%%%%%%%%%%%%%%%%%%%%%%%%%%%%%%%%%%%%%%%%%%%%%%
\large\textbf{Oreofe Solarin} \hfill \textbf{Problem Set - 3}   \\
Email: ons8@case.edu \hfill ID: 123456789 \\
\normalsize Course: CSDS 337 - Compiler Design \hfill Term: Spring 2024\\
Instructor: Dr. Vipin Chaudhary \hfill Due Date: $28^{th}$ Feb, 2024 \\ \\
Number of hours delay for this Problem Set: \hfill\\
Cumulative number of hours delay so far: \hfill 3 \\ \\
I discussed this homework with: \hfill  \\ \\
%\underline{\bf SUBMISSION GUIDELINES:} Submit a zip file that includes the %written answers and the flex file for Problem 4. \\

\noindent\rule{7in}{2.8pt}
%%%%%%%%%%%%%%%%%%%%%%%%%%%%%%%%%%%%%%%%%%%%%%%%%%%%%%%%%%%%%%%%%%%%%%%%%%%%%%%%%%%%%%%%%%%%%%%%%%%%%%%%%%%%%%%%%%%%%%%%%%%%%%%%%%%%%%%%
% Problem 1
%%%%%%%%%%%%%%%%%%%%%%%%%%%%%%%%%%%%%%%%%%%%%%%%%%%%%%%%%%%%%%%%%%%%%%%%%%%%%%%%%%%%%%%%%%%%%%%%%%%%%%%%%%%%%%%%%%%%%%%%%%%%%%%%%%%%%%%%
\begin{problem}{1 - 10 points}
Design grammar for the following language:

\begin{itemize}
    \item The set of all strings of 0s and 1s such that every 0 is immediately followed by at least one 1.
\end{itemize}

\end{problem}
\begin{solution}
\begin{align*}
S & \rightarrow 1S \,|\, 01S \,|\, \varepsilon
\end{align*}
\end{solution} 
\noindent\rule{7in}{2.8pt}

%%%%%%%%%%%%%%%%%%%%%%%%%%%%%%%%%%%%%%%%%%%%%%%%%%%%%%%%%%%%%%%%%%%%%%%%%
% Problem 2
%%%%%%%%%%%%%%%%%%%%%%%%%%%%%%%%%%%%%%%%%%%%%%%%%%%%%%%%%%%%%%%%%%%%%%%%%%%%%%%%%%%%%%%%%%%%%%%%%%%%%%%%%%%%%%%%%%%%%%%%%%%%%%%%%%%%%%%%

\begin{problem}{2 - 20 points}
The following is a grammar for regular expressions over symbols $a$ and $b$ only, using $+$ in place of $|$ for union, to avoid conflict with the use  of vertical bar as a metasymbol in grammars:  \\

\noindent $rexpr \rightarrow rexpr + rterm \quad | \quad rterm  \\
rterm \rightarrow rterm  \quad rfactor \quad | \quad rfactor  \\
rfactor \rightarrow rfactor * \quad | \quad rprimary  \\
rprimary \rightarrow a | b$ \\

\begin{enumerate}[a]
    \item Left factor this grammar.
    \item Does left factoring make the grammar suitable for top-down parsing?
    \item In addition to left factoring, eliminate left recursion from the original  grammar.
    \item Is the resulting grammar suitable for top-down parsing?
\end{enumerate}

\end{problem}
\begin{solution}
\begin{enumerate}[a]

\item

We can use the common prefixes:
- For `rexpr`: `rexpr` 
- For `rterm`: `rterm`
- For `rfactor`: `rfactor`

So, the left factored grammar becomes:

\begin{align*}
rexpr &\rightarrow rterm \ rexpr' \\
rexpr' &\rightarrow + \ rterm \ rexpr' \quad | \quad \varepsilon \\
rterm &\rightarrow rfactor \ rterm' \\
rterm' &\rightarrow rfactor \ rterm' \quad | \quad \varepsilon \\
rfactor &\rightarrow rprimary \ rfactor' \\
rfactor' &\rightarrow * \ rfactor' \quad | \quad \varepsilon \\
rprimary &\rightarrow a \quad | \quad b
\end{align*}

\item Left factoring does not necessarily make a grammar suitable for top-down parsing. Left factoring  helps to avoid ambiguity and reduce the number of alternatives, it doesn't inherently ensure suitability for top-down parsing. The suitability for top-down parsing also depends on whether the grammar is left recursive or not, among other factors.

\item We can rewrite productions to remove direct or indirect left recursion. The given grammar has a indirect left recursion:



Rewriting the grammar to remove this indirect left recursion. 

\begin{align*}
rexpr &\rightarrow rterm \ rexpr' \\
rexpr' &\rightarrow + \ rterm \ rexpr' \quad | \quad \varepsilon \\
rterm &\rightarrow rfactor \ rterm' \\
rterm' &\rightarrow rfactor \ rterm' \quad | \quad \varepsilon \\
rfactor &\rightarrow rprimary \ rfactor' \\
rfactor' &\rightarrow * \ rfactor' \quad | \quad \varepsilon \\
rprimary &\rightarrow a \quad | \quad b
\end{align*}


d. The resulting grammar is suitable for top-down parsing. It's non-ambiguous, non-left recursive, and has a clear start symbol.

\end{enumerate}
\end{solution} 
\noindent\rule{7in}{2.8pt}

%%%%%%%%%%%%%%%%%%%%%%%%%%%%%%%%%%%%%%%%%%%%%%%%%%%%%%%%%%%%%%%%%%%%%%%%%
% Problem 3
%%%%%%%%%%%%%%%%%%%%%%%%%%%%%%%%%%%%%%%%%%%%%%%%%%%%%%%%%%%%%%%%%%%%%%%%%%%%%%%%%%%%%%%%%%%%%%%%%%%%%%%%%%%%%%%%%%%%%%%%%%%%%%%%%%%%%%%%

\begin{problem}{3 - 20 points}
Consider the grammar for $S  \longrightarrow S + S | SS | (S) | S * | a $ and the string $(a + a)* a$. 

\begin{enumerate}[a]
    \item Devise a predictive parser and show the parsing tables. You may use left-factor and/or eliminate left-recursion from your grammar first.
    \item Compute FIRST and FOLLOW for your grammar. 
\end{enumerate}

\end{problem}

\begin{solution}
\begin{enumerate}[a]
    \item 
Left-factored Grammar:
\[ S \longrightarrow aS' \ | \ (S)S' \]
\[ S' \longrightarrow +S \ | \ \epsilon \ | \ *S \]

Parsing table:

\[
\begin{array}{|c|c|c|c|c|c|c|}
\hline
& a & ( & ) & + & * & \$ \\
\hline
S & S \longrightarrow aS' & S \longrightarrow (S)S' & & & & \\
\hline
S' & & & S' \longrightarrow \epsilon & S' \longrightarrow +S & S' \longrightarrow *S & S' \longrightarrow \epsilon \\
\hline
\end{array}
\]

We can construct the predictive parser using the parsering table above

\begin{enumerate}
    \item Initialize stack with start symbol $S$ and input string.
    \item Repeat until the stack is empty:
    \begin{enumerate}[a]
        \item If the top of the stack is a terminal symbol, match it with the input symbol and pop both.
        \item If the top of the stack is a non-terminal symbol:
        \begin{enumerate}
            \item Look up the corresponding entry in the parsing table.
            \item Replace the non-terminal symbol on the stack with the production from the parsing table.
        \end{enumerate}
    \end{enumerate}
    \item If at the end, the stack is empty, and the input string is fully parsed, accept. Otherwise, reject.
\end{enumerate}


 
    \item For the FIRST sets:

\[ FIRST(S) = \{a, (\} \]
\[ FIRST(S') = \{+, *, \epsilon\} \]

For the FOLLOW sets:

\[ FOLLOW(S) = \{\$, ), +, *\} \]
\[ FOLLOW(S') = \{\$, ), +, *\} \]
\end{enumerate}
\end{solution} 

\noindent\rule{7in}{2.8pt}
%%%%%%%%%%%%%%%%%%%%%%%%%%%%%%%%%%%%%%%%%%%%%%%%%%%%%%%%%%%%%%%%%%%%%%%%%
%%%%%%%%%%%%%%%%%%%%%%%%%%%%%%%%%%%%%%%%%%%%%%%%%%%%%%%%%%%%%%%%%%%%%%%%%
% Problem 4
%%%%%%%%%%%%%%%%%%%%%%%%%%%%%%%%%%%%%%%%%%%%%%%%%%%%%%%%%%%%%%%%%%%%%%%%%%%%%%%%%%%%%%%%%%%%%%%%%%%%%%%%%%%%%%%%%%%%%%%%%%%%%%%%%%%%%%%%

\begin{problem}{4 - 10 points}
Give the bottom-up parses for the following input string and grammar: $aaa*a++$ and $S  \longrightarrow SS+ | SS* | a $. 

\end{problem}

\begin{solution}
We start with an empty stack and the input string to the left of the pointer.

\noindent Step 1:\\
\quad Stack: [ ]\\
\quad Remaining input: $aaa*a++$ \\

\noindent Step 2:\\
\quad Apply a shift operation for the first three 'a's.\\
\quad Stack: [aaa]\\
\quad Remaining input: $*a++$ \\

\noindent Step 3:\\
\quad Apply a shift operation for '*'.\\
\quad Stack: [aaa, *]\\
\quad Remaining input: $a++$ \\

\noindent Step 4:\\
\quad Apply a shift operation for 'a'.\\
\quad Stack: [aaa, *a]\\
\quad Remaining input: $++$ \\

\noindent Step 5:\\
\quad Apply a reduce operation by replacing 'a' with $S$.\\
\quad Stack: [aaa, $S$]\\
\quad Remaining input: $++$ \\

\noindent Step 6:\\
\quad Apply a reduce operation by replacing '$S$*$S$' with $S$.\\
\quad Stack: [$S$]\\
\quad Remaining input: $++$ \\ 

\noindent Step 7:\\
\quad Apply a shift operation for '+'.\\
\quad Stack: [$S$, +]\\
\quad Remaining input: $+$ \\

\noindent Step 8:\\
\quad Apply a shift operation for '+'.\\
\quad Stack: [$S$, +, +]\\
\quad Remaining input:  \\

\noindent Step 9:\\
\quad Apply a reduce operation by replacing '+' with $S$.\\
\quad Stack: [$S$, $S$]\\
\quad Remaining input:  \\

\noindent Step 10:\\
\quad Apply a reduce operation by replacing '$S$+$S$' with $S$.\\
\quad Stack: [$S$]\\
\quad Remaining input:  \\

\noindent Now, the stack contains only the start symbol $S$ and there's no remaining input. So, the bottom-up parse is successful, and the given input string is parsed by the grammar.
\end{solution}

%%%%%%%%%%%%%%%%%%%%%%%%%%%%%%%%%%%%%%%%%%%%%%%%%%%%%%%%%%%%%%%%%%%%%%%%%
%%%%%%%%%%%%%%%%%%%%%%%%%%%%%%%%%%%%%%%%%%%%%%%%%%%%%%%%%%%%%%%%%%%%%%%%%
% Problem 5
%%%%%%%%%%%%%%%%%%%%%%%%%%%%%%%%%%%%%%%%%%%%%%%%%%%%%%%%%%%%%%%%%%%%%%%%%%%%%%%%%%%%%%%%%%%%%%%%%%%%%%%%%%%%%%%%%%%%%%%%%%%%%%%%%%%%%%%%

\begin{problem}{5 - 20 points}
Construct the SLR sets of items for the (augmented) grammar $S  \longrightarrow SS+ | SS* | a $. Compute the GOTO function for these sets of items. Show the parsing table for this grammar. Is this grammar SLR?

\end{problem}

\begin{solution}
\noindent LR(0) items:
\begin{align*}
S' & \rightarrow \cdot S \\
S & \rightarrow \cdot SS+ \\
S & \rightarrow \cdot SS* \\
S & \rightarrow \cdot a \\
S & \rightarrow S \cdot S+ \\
S & \rightarrow S \cdot S* \\
S & \rightarrow S \cdot a \\
\end{align*}

\noindent The GOTO function items:

\begin{align*}
GOTO(0, S) &= 1 \\
GOTO(0, a) &= 3 \\
GOTO(1, S) &= 2 \\
GOTO(1, a) &= 3 \\
GOTO(2, +) &= 4 \\
GOTO(2, *) &= 5 \\
GOTO(2, \$) &= \text{accept} \\
GOTO(3, S) &= 6 \\
GOTO(3, a) &= 3 \\
GOTO(4, S) &= 7 \\
GOTO(4, a) &= 3 \\
GOTO(5, S) &= 8 \\
GOTO(5, a) &= 3 \\
GOTO(6, +) &= 4 \\
GOTO(6, *) &= 5 \\
GOTO(6, \$) &= \text{accept} \\
GOTO(7, +) &= 4 \\
GOTO(7, *) &= 5 \\
GOTO(7, \$) &= \text{accept} \\
GOTO(8, +) &= 4 \\
GOTO(8, *) &= 5 \\
GOTO(8, \$) &= \text{accept} \\
\end{align*}


\noindent Parsing table:
\[\begin{array}{|c|c|c|c|c|}
\hline
\text{State} & a & + & * & \$ \\
\hline
0 & s3 & & & \\
1 & s3 & & & \\
2 &  & s4 & s5 & accept \\
3 & s3 & & & r3 \\
4 & s3 & & & r1 \\
5 & s3 & & & r2 \\
6 & s3 & s4 & s5 & \\
7 & s3 & r1 & r1 & \\
8 & s3 & r2 & r2 & \\
\hline
\end{array}
\]

\noindent "s" stands for shift, "r" stands for reduce, and "accept".

\noindent Looking at the table, we can see that there are no conflicts, so the grammar is SLR(1).

\noindent Thus, we have successfully constructed the SLR sets of items, computed the GOTO function, showed the parsing table, and determined that the grammar is SLR.
\end{solution} 

%%%%%%%%%%%%%%%%%%%%%%%%%%%%%%%%%%%%%%%%%%%%%%%%%%%%%%%%%%%%%%%%%%%%%%%%%
%%%%%%%%%%%%%%%%%%%%%%%%%%%%%%%%%%%%%%%%%%%%%%%%%%%%%%%%%%%%%%%%%%%%%%%%%
% Problem 6
%%%%%%%%%%%%%%%%%%%%%%%%%%%%%%%%%%%%%%%%%%%%%%%%%%%%%%%%%%%%%%%%%%%%%%%%%%%%%%%%%%%%%%%%%%%%%%%%%%%%%%%%%%%%%%%%%%%%%%%%%%%%%%%%%%%%%%%%

\begin{problem}{6 - 20 points}
Construct the canonical parsing table for the following augmented grammar:

\noindent $S'  \longrightarrow S \\
S \longrightarrow AA \\ 
A \longrightarrow a A | b $\\

\end{problem}

\begin{solution}

\noindent LR(0) items:

\begin{enumerate}
    \item  \(\textbf{S'} \longrightarrow \cdot \textbf{S}\)
    \item \(\textbf{S} \longrightarrow \cdot \textbf{AA}\)
    \item \(\textbf{S} \longrightarrow \textbf{A} \cdot \textbf{A}\)
    \item \(\textbf{A} \longrightarrow \cdot \textbf{aA}\)
    \item  \(\textbf{A} \longrightarrow \cdot \textbf{b}\)
\end{enumerate}

\noindent The parsing table using these LR(0) items:

\noindent States (Closure of LR(0) items):

\noindent \textbf{State 0}: \(\textbf{S'} \longrightarrow \cdot \textbf{S}\) \\

\noindent \textbf{State 1}:
\begin{enumerate}
\item 
\(\textbf{S'} \longrightarrow \textbf{S} \cdot\)
    \item 
\(\textbf{S} \longrightarrow \cdot \textbf{AA}\)
    \item 
\(\textbf{A} \longrightarrow \cdot \textbf{aA}\)
    \item 
\(\textbf{A} \longrightarrow \cdot \textbf{b}\)
    
\end{enumerate} 
\noindent \textbf{State 2}:
\begin{enumerate}
    \item 
\(\textbf{S} \longrightarrow \textbf{A} \cdot \textbf{A}\)
    \item 
\(\textbf{A} \longrightarrow \cdot \textbf{aA}\)
    \item 
\(\textbf{A} \longrightarrow \cdot \textbf{b}\)
    
\end{enumerate}
\noindent \textbf{State 3}:  \(\textbf{S} \longrightarrow \textbf{AA} \cdot\) \\
\noindent  \textbf{State 4}: \(\textbf{A} \longrightarrow \textbf{a} \cdot \textbf{A}\) \\
\noindent  \textbf{State 5}: \(\textbf{A} \longrightarrow \textbf{b} \cdot\)


\begin{center}
\begin{tabular}{|c|c|c|c|c|c|}
\hline
State & \(\textbf{a}\) & \(\textbf{b}\) & \(\textbf{S}\) & \(\textbf{A}\) & \(\textbf{\$}\) \\
\hline
0 &  &  & 1 &  &  \\
1 &  &  &  & 2 & \textbf{Acc} \\
2 & 4 & 5 &  &  &  \\
3 &  &  &  &  &  \\
4 &  &  &  &  &  \\
5 &  &  &  &  &  \\
\hline
\end{tabular}
\end{center}

\noindent Acc denotes accepting the input

\end{solution}

\noindent\rule{7in}{2.8pt}

\end{document}
 